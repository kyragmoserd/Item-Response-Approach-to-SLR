% Options for packages loaded elsewhere
\PassOptionsToPackage{unicode}{hyperref}
\PassOptionsToPackage{hyphens}{url}
%
\documentclass[
]{article}
\usepackage{lmodern}
\usepackage{amssymb,amsmath}
\usepackage{ifxetex,ifluatex}
\ifnum 0\ifxetex 1\fi\ifluatex 1\fi=0 % if pdftex
  \usepackage[T1]{fontenc}
  \usepackage[utf8]{inputenc}
  \usepackage{textcomp} % provide euro and other symbols
\else % if luatex or xetex
  \usepackage{unicode-math}
  \defaultfontfeatures{Scale=MatchLowercase}
  \defaultfontfeatures[\rmfamily]{Ligatures=TeX,Scale=1}
\fi
% Use upquote if available, for straight quotes in verbatim environments
\IfFileExists{upquote.sty}{\usepackage{upquote}}{}
\IfFileExists{microtype.sty}{% use microtype if available
  \usepackage[]{microtype}
  \UseMicrotypeSet[protrusion]{basicmath} % disable protrusion for tt fonts
}{}
\makeatletter
\@ifundefined{KOMAClassName}{% if non-KOMA class
  \IfFileExists{parskip.sty}{%
    \usepackage{parskip}
  }{% else
    \setlength{\parindent}{0pt}
    \setlength{\parskip}{6pt plus 2pt minus 1pt}}
}{% if KOMA class
  \KOMAoptions{parskip=half}}
\makeatother
\usepackage{xcolor}
\IfFileExists{xurl.sty}{\usepackage{xurl}}{} % add URL line breaks if available
\IfFileExists{bookmark.sty}{\usepackage{bookmark}}{\usepackage{hyperref}}
\hypersetup{
  pdftitle={Latent trait analysis},
  hidelinks,
  pdfcreator={LaTeX via pandoc}}
\urlstyle{same} % disable monospaced font for URLs
\usepackage[margin=1in]{geometry}
\usepackage{color}
\usepackage{fancyvrb}
\newcommand{\VerbBar}{|}
\newcommand{\VERB}{\Verb[commandchars=\\\{\}]}
\DefineVerbatimEnvironment{Highlighting}{Verbatim}{commandchars=\\\{\}}
% Add ',fontsize=\small' for more characters per line
\usepackage{framed}
\definecolor{shadecolor}{RGB}{248,248,248}
\newenvironment{Shaded}{\begin{snugshade}}{\end{snugshade}}
\newcommand{\AlertTok}[1]{\textcolor[rgb]{0.94,0.16,0.16}{#1}}
\newcommand{\AnnotationTok}[1]{\textcolor[rgb]{0.56,0.35,0.01}{\textbf{\textit{#1}}}}
\newcommand{\AttributeTok}[1]{\textcolor[rgb]{0.77,0.63,0.00}{#1}}
\newcommand{\BaseNTok}[1]{\textcolor[rgb]{0.00,0.00,0.81}{#1}}
\newcommand{\BuiltInTok}[1]{#1}
\newcommand{\CharTok}[1]{\textcolor[rgb]{0.31,0.60,0.02}{#1}}
\newcommand{\CommentTok}[1]{\textcolor[rgb]{0.56,0.35,0.01}{\textit{#1}}}
\newcommand{\CommentVarTok}[1]{\textcolor[rgb]{0.56,0.35,0.01}{\textbf{\textit{#1}}}}
\newcommand{\ConstantTok}[1]{\textcolor[rgb]{0.00,0.00,0.00}{#1}}
\newcommand{\ControlFlowTok}[1]{\textcolor[rgb]{0.13,0.29,0.53}{\textbf{#1}}}
\newcommand{\DataTypeTok}[1]{\textcolor[rgb]{0.13,0.29,0.53}{#1}}
\newcommand{\DecValTok}[1]{\textcolor[rgb]{0.00,0.00,0.81}{#1}}
\newcommand{\DocumentationTok}[1]{\textcolor[rgb]{0.56,0.35,0.01}{\textbf{\textit{#1}}}}
\newcommand{\ErrorTok}[1]{\textcolor[rgb]{0.64,0.00,0.00}{\textbf{#1}}}
\newcommand{\ExtensionTok}[1]{#1}
\newcommand{\FloatTok}[1]{\textcolor[rgb]{0.00,0.00,0.81}{#1}}
\newcommand{\FunctionTok}[1]{\textcolor[rgb]{0.00,0.00,0.00}{#1}}
\newcommand{\ImportTok}[1]{#1}
\newcommand{\InformationTok}[1]{\textcolor[rgb]{0.56,0.35,0.01}{\textbf{\textit{#1}}}}
\newcommand{\KeywordTok}[1]{\textcolor[rgb]{0.13,0.29,0.53}{\textbf{#1}}}
\newcommand{\NormalTok}[1]{#1}
\newcommand{\OperatorTok}[1]{\textcolor[rgb]{0.81,0.36,0.00}{\textbf{#1}}}
\newcommand{\OtherTok}[1]{\textcolor[rgb]{0.56,0.35,0.01}{#1}}
\newcommand{\PreprocessorTok}[1]{\textcolor[rgb]{0.56,0.35,0.01}{\textit{#1}}}
\newcommand{\RegionMarkerTok}[1]{#1}
\newcommand{\SpecialCharTok}[1]{\textcolor[rgb]{0.00,0.00,0.00}{#1}}
\newcommand{\SpecialStringTok}[1]{\textcolor[rgb]{0.31,0.60,0.02}{#1}}
\newcommand{\StringTok}[1]{\textcolor[rgb]{0.31,0.60,0.02}{#1}}
\newcommand{\VariableTok}[1]{\textcolor[rgb]{0.00,0.00,0.00}{#1}}
\newcommand{\VerbatimStringTok}[1]{\textcolor[rgb]{0.31,0.60,0.02}{#1}}
\newcommand{\WarningTok}[1]{\textcolor[rgb]{0.56,0.35,0.01}{\textbf{\textit{#1}}}}
\usepackage{graphicx}
\makeatletter
\def\maxwidth{\ifdim\Gin@nat@width>\linewidth\linewidth\else\Gin@nat@width\fi}
\def\maxheight{\ifdim\Gin@nat@height>\textheight\textheight\else\Gin@nat@height\fi}
\makeatother
% Scale images if necessary, so that they will not overflow the page
% margins by default, and it is still possible to overwrite the defaults
% using explicit options in \includegraphics[width, height, ...]{}
\setkeys{Gin}{width=\maxwidth,height=\maxheight,keepaspectratio}
% Set default figure placement to htbp
\makeatletter
\def\fps@figure{htbp}
\makeatother
\setlength{\emergencystretch}{3em} % prevent overfull lines
\providecommand{\tightlist}{%
  \setlength{\itemsep}{0pt}\setlength{\parskip}{0pt}}
\setcounter{secnumdepth}{-\maxdimen} % remove section numbering
\ifluatex
  \usepackage{selnolig}  % disable illegal ligatures
\fi

\title{Latent trait analysis}
\author{}
\date{\vspace{-2.5em}}

\begin{document}
\maketitle

FOR EVERYTHING BELOW, I DROPPED OUT THE THREE ``OTHER'' RESPONSE OPTIONS
FROM THE INCIDENCE MATRICES

\#ignore this chunk, it just loads and builds network object

\begin{Shaded}
\begin{Highlighting}[]
\NormalTok{packs =}\KeywordTok{c}\NormalTok{(}\StringTok{\textquotesingle{}tidyverse\textquotesingle{}}\NormalTok{,}\StringTok{\textquotesingle{}purrr\textquotesingle{}}\NormalTok{,}\StringTok{\textquotesingle{}data.table\textquotesingle{}}\NormalTok{,}\StringTok{\textquotesingle{}statnet\textquotesingle{}}\NormalTok{,}\StringTok{\textquotesingle{}latentnet\textquotesingle{}}\NormalTok{,}\StringTok{\textquotesingle{}bipartite\textquotesingle{}}\NormalTok{,}\StringTok{\textquotesingle{}lvm4net\textquotesingle{}}\NormalTok{,}
         \StringTok{\textquotesingle{}ggthemes\textquotesingle{}}\NormalTok{,}\StringTok{\textquotesingle{}here\textquotesingle{}}\NormalTok{,}\StringTok{\textquotesingle{}ggnetwork\textquotesingle{}}\NormalTok{,}\StringTok{\textquotesingle{}gridExtra\textquotesingle{}}\NormalTok{,}\StringTok{\textquotesingle{}ggrepel\textquotesingle{}}\NormalTok{)}
\NormalTok{need =}\StringTok{ }\NormalTok{packs[}\OperatorTok{!}\KeywordTok{sapply}\NormalTok{(packs,}\ControlFlowTok{function}\NormalTok{(x) }\KeywordTok{suppressMessages}\NormalTok{(}\KeywordTok{require}\NormalTok{(x,}\DataTypeTok{character.only=}\NormalTok{T)))]}
\KeywordTok{sapply}\NormalTok{(need,}\ControlFlowTok{function}\NormalTok{(x) }\KeywordTok{suppressMessages}\NormalTok{(}\KeywordTok{install.packages}\NormalTok{(x,,}\DataTypeTok{type=} \StringTok{\textquotesingle{}source\textquotesingle{}}\NormalTok{)))}
\end{Highlighting}
\end{Shaded}

\begin{verbatim}
## named list()
\end{verbatim}

\begin{Shaded}
\begin{Highlighting}[]
\KeywordTok{sapply}\NormalTok{(packs[need],}\ControlFlowTok{function}\NormalTok{(x) }\KeywordTok{suppressMessages}\NormalTok{(}\KeywordTok{require}\NormalTok{(x,}\DataTypeTok{character.only=}\NormalTok{T)))}
\end{Highlighting}
\end{Shaded}

\begin{verbatim}
## named list()
\end{verbatim}

\begin{Shaded}
\begin{Highlighting}[]
\KeywordTok{library}\NormalTok{(readxl)}
\NormalTok{orig =}\StringTok{ }\NormalTok{readxl}\OperatorTok{::}\KeywordTok{read\_excel}\NormalTok{(}\StringTok{\textquotesingle{}input/SLRSurvey\_Full.xlsx\textquotesingle{}}\NormalTok{)}
\NormalTok{orig}\OperatorTok{$}\NormalTok{Q4[}\KeywordTok{is.na}\NormalTok{(orig}\OperatorTok{$}\NormalTok{Q4)]\textless{}{-}}\StringTok{\textquotesingle{}Other\textquotesingle{}}
\CommentTok{\#recode anything with fewer than 10 respondents as other}
\CommentTok{\# see what wed recode}
\CommentTok{\#as.data.table(table(orig$Q4))[order({-}N),][N\textless{}10,]}
\NormalTok{orig}\OperatorTok{$}\NormalTok{Q4[orig}\OperatorTok{$}\NormalTok{Q4 }\OperatorTok{\%in\%}\StringTok{ }\KeywordTok{as.data.table}\NormalTok{(}\KeywordTok{table}\NormalTok{(orig}\OperatorTok{$}\NormalTok{Q4))[}\KeywordTok{order}\NormalTok{(}\OperatorTok{{-}}\NormalTok{N),][N}\OperatorTok{\textless{}}\DecValTok{10}\NormalTok{,]}\OperatorTok{$}\NormalTok{V1] \textless{}{-}}\StringTok{ \textquotesingle{}Other\textquotesingle{}}
\NormalTok{orig}\OperatorTok{$}\NormalTok{Q4[}\KeywordTok{grepl}\NormalTok{(}\StringTok{\textquotesingle{}Other\textquotesingle{}}\NormalTok{,orig}\OperatorTok{$}\NormalTok{Q4)]\textless{}{-}}\StringTok{\textquotesingle{}Other\textquotesingle{}}
\NormalTok{incidence\_dt =}\StringTok{ }\KeywordTok{fread}\NormalTok{(}\StringTok{"CurrentFiles/Data/AdjMatrix\_MinOtherRecode.csv"}\NormalTok{)}

\NormalTok{incidence\_mat =}\StringTok{ }\KeywordTok{as.matrix}\NormalTok{(incidence\_dt[,}\OperatorTok{{-}}\KeywordTok{c}\NormalTok{(}\StringTok{\textquotesingle{}ResponseId\textquotesingle{}}\NormalTok{,}\StringTok{\textquotesingle{}DK\textquotesingle{}}\NormalTok{)])}
\KeywordTok{rownames}\NormalTok{(incidence\_mat)\textless{}{-}incidence\_dt}\OperatorTok{$}\NormalTok{ResponseId}
\CommentTok{\#drop isolates}
\NormalTok{incidence\_mat =}\StringTok{ }\NormalTok{incidence\_mat[}\KeywordTok{rowSums}\NormalTok{(incidence\_mat)}\OperatorTok{!=}\DecValTok{0}\NormalTok{,]}
\NormalTok{incidence\_mat =}\StringTok{ }\NormalTok{incidence\_mat[,}\OperatorTok{!}\KeywordTok{grepl}\NormalTok{(}\StringTok{\textquotesingle{}Other\textquotesingle{}}\NormalTok{,}\KeywordTok{colnames}\NormalTok{(incidence\_mat))]}



\CommentTok{\#create network object}
\NormalTok{bip\_net =}\StringTok{ }\KeywordTok{as.network}\NormalTok{(incidence\_mat,}\DataTypeTok{matrix.type =} \StringTok{\textquotesingle{}incidence\textquotesingle{}}\NormalTok{,}\DataTypeTok{bipartite =}\NormalTok{ T,}\DataTypeTok{directed =}\NormalTok{ F,}\DataTypeTok{loops =}\NormalTok{ F)}
\CommentTok{\#code actor types}
\NormalTok{bip\_net }\OperatorTok{\%v\%}\StringTok{ \textquotesingle{}Actor\_Type\textquotesingle{}}\NormalTok{ \textless{}{-}}\StringTok{ }\NormalTok{orig}\OperatorTok{$}\NormalTok{Q4[}\KeywordTok{match}\NormalTok{(}\KeywordTok{network.vertex.names}\NormalTok{(bip\_net),orig}\OperatorTok{$}\NormalTok{ResponseId)]}
\CommentTok{\#code concept types}
\NormalTok{concept\_types =}\StringTok{ }\KeywordTok{fread}\NormalTok{(}\StringTok{\textquotesingle{}input/Combined\_VectorTypes\_NoNewOther.csv\textquotesingle{}}\NormalTok{)}
\NormalTok{bip\_net }\OperatorTok{\%v\%}\StringTok{ \textquotesingle{}Concept\_Type\textquotesingle{}}\NormalTok{ \textless{}{-}}\StringTok{ }\NormalTok{concept\_types}\OperatorTok{$}\NormalTok{Type[}\KeywordTok{match}\NormalTok{(}\KeywordTok{network.vertex.names}\NormalTok{(bip\_net), concept\_types}\OperatorTok{$}\NormalTok{Vector)]}
\KeywordTok{set.vertex.attribute}\NormalTok{(bip\_net,}\StringTok{\textquotesingle{}Concept\_Type\textquotesingle{}}\NormalTok{,}\DataTypeTok{value =} \StringTok{\textquotesingle{}Person\textquotesingle{}}\NormalTok{,}\DataTypeTok{v =} \KeywordTok{which}\NormalTok{(}\KeywordTok{is.na}\NormalTok{(bip\_net }\OperatorTok{\%v\%}\StringTok{ \textquotesingle{}Concept\_Type\textquotesingle{}}\NormalTok{)))}

\NormalTok{bip\_net }\OperatorTok{\%v\%}\StringTok{ \textquotesingle{}id\textquotesingle{}}\NormalTok{ \textless{}{-}}\StringTok{ }\KeywordTok{network.vertex.names}\NormalTok{(bip\_net)}
\NormalTok{bip\_net }\OperatorTok{\%v\%}\StringTok{ \textquotesingle{}id\_concept\textquotesingle{}}\NormalTok{ \textless{}{-}}\StringTok{ }\KeywordTok{ifelse}\NormalTok{(\{bip\_net }\OperatorTok{\%v\%}\StringTok{ \textquotesingle{}Concept\_Type\textquotesingle{}}\NormalTok{\} }\OperatorTok{==}\StringTok{ \textquotesingle{}Person\textquotesingle{}}\NormalTok{,}\KeywordTok{network.vertex.names}\NormalTok{(bip\_net),bip\_net }\OperatorTok{\%v\%}\StringTok{ \textquotesingle{}Concept\_Type\textquotesingle{}}\NormalTok{)}

\NormalTok{bip\_net }\OperatorTok{\%v\%}\StringTok{ \textquotesingle{}b1\_dummy\_b2\_names\textquotesingle{}}\NormalTok{ \textless{}{-}}\StringTok{ }\KeywordTok{ifelse}\NormalTok{(\{bip\_net }\OperatorTok{\%v\%}\StringTok{ \textquotesingle{}Concept\_Type\textquotesingle{}}\NormalTok{\} }\OperatorTok{==}\StringTok{ \textquotesingle{}Person\textquotesingle{}}\NormalTok{,}\StringTok{\textquotesingle{}Person\textquotesingle{}}\NormalTok{,bip\_net }\OperatorTok{\%v\%}\StringTok{ \textquotesingle{}vertex.names\textquotesingle{}}\NormalTok{)}
\end{Highlighting}
\end{Shaded}

\begin{Shaded}
\begin{Highlighting}[]
\KeywordTok{require}\NormalTok{(corrplot)}
\end{Highlighting}
\end{Shaded}

\begin{verbatim}
## Loading required package: corrplot
\end{verbatim}

\begin{verbatim}
## corrplot 0.84 loaded
\end{verbatim}

\begin{Shaded}
\begin{Highlighting}[]
\CommentTok{\#convert to incidence matrix}
\NormalTok{Y =}\StringTok{ }\KeywordTok{as.sociomatrix}\NormalTok{(bip\_net)}
\NormalTok{Y =}\StringTok{ }\NormalTok{Y[,}\OperatorTok{!}\KeywordTok{grepl}\NormalTok{(}\StringTok{\textquotesingle{}Other\textquotesingle{}}\NormalTok{,}\KeywordTok{colnames}\NormalTok{(Y))]}
\KeywordTok{data.table}\NormalTok{(Y)}
\end{Highlighting}
\end{Shaded}

\begin{verbatim}
##      SLR Plan Vulnerable Local Tax Lobby Permits Info Platform DACs Green Infra
##   1:        1          0         0     1       0             0    0           0
##   2:        1          0         0     0       0             0    0           1
##   3:        0          0         0     0       0             1    0           0
##   4:        1          0         0     0       0             1    0           0
##   5:        1          1         0     0       0             1    0           0
##  ---                                                                           
## 694:        0          1         1     0       0             0    0           0
## 695:        1          0         0     0       0             0    0           1
## 696:        1          0         1     1       0             0    0           0
## 697:        0          0         1     0       0             0    0           1
## 698:        1          0         0     0       0             0    0           1
##      Vision Innov Design Local Response Regional Authority Existing Agency
##   1:      0            0              0                  0               1
##   2:      0            0              0                  0               0
##   3:      0            1              0                  0               0
##   4:      0            0              0                  0               1
##   5:      0            0              0                  0               0
##  ---                                                                      
## 694:      0            0              0                  1               0
## 695:      0            1              0                  0               0
## 696:      0            0              0                  0               0
## 697:      0            1              0                  0               0
## 698:      0            0              0                  0               0
##      Collab Transpo Water Stormwater Energy Ecosystem Erosion Commercial DACsC
##   1:      0       1     1          0      0         0       0          0     1
##   2:      1       1     0          0      0         1       0          0     1
##   3:      0       0     1          1      0         1       0          0     0
##   4:      0       1     0          0      0         0       0          0     1
##   5:      0       1     0          0      0         0       0          0     0
##  ---                                                                          
## 694:      0       0     0          0      0         1       1          0     1
## 695:      0       0     1          1      0         1       0          0     0
## 696:      0       1     1          0      0         0       1          0     0
## 697:      0       1     0          1      0         1       0          0     0
## 698:      0       0     1          1      0         1       0          0     0
##      Econ Growth Property Value Housing Public Health Human Financial
##   1:           0              0       0             0     0         1
##   2:           0              0       0             0     0         0
##   3:           0              0       0             0     0         0
##   4:           0              0       1             0     0         0
##   5:           1              0       1             0     0         1
##  ---                                                                 
## 694:           0              0       0             0     0         0
## 695:           0              0       0             0     0         0
## 696:           0              0       0             0     0         1
## 697:           0              0       0             0     0         0
## 698:           0              0       0             0     1         0
##      Collab Experience Stakeholder Opposition Political Leader SLR Uncertainty
##   1:                 0                      0                0               0
##   2:                 0                      0                1               0
##   3:                 1                      1                0               0
##   4:                 0                      0                0               0
##   5:                 0                      0                1               0
##  ---                                                                          
## 694:                 0                      0                0               0
## 695:                 0                      1                0               0
## 696:                 0                      0                0               0
## 697:                 0                      0                1               1
## 698:                 0                      1                1               0
##      Sci Info Org Leader Overall Plan PermitsB Public Support CBO Relations
##   1:        0          1            0        0              0             0
##   2:        0          0            1        0              1             0
##   3:        0          0            0        0              0             0
##   4:        0          0            0        0              0             0
##   5:        0          0            1        0              0             0
##  ---                                                                       
## 694:        0          0            0        0              0             0
## 695:        0          0            1        0              1             0
## 696:        0          1            0        1              0             0
## 697:        0          0            1        0              0             0
## 698:        0          0            0        0              0             0
\end{verbatim}

\begin{Shaded}
\begin{Highlighting}[]
\KeywordTok{colnames}\NormalTok{(Y)}
\end{Highlighting}
\end{Shaded}

\begin{verbatim}
##  [1] "SLR Plan"               "Vulnerable"             "Local Tax"             
##  [4] "Lobby"                  "Permits"                "Info Platform"         
##  [7] "DACs"                   "Green Infra"            "Vision"                
## [10] "Innov Design"           "Local Response"         "Regional Authority"    
## [13] "Existing Agency"        "Collab"                 "Transpo"               
## [16] "Water"                  "Stormwater"             "Energy"                
## [19] "Ecosystem"              "Erosion"                "Commercial"            
## [22] "DACsC"                  "Econ Growth"            "Property Value"        
## [25] "Housing"                "Public Health"          "Human"                 
## [28] "Financial"              "Collab Experience"      "Stakeholder Opposition"
## [31] "Political Leader"       "SLR Uncertainty"        "Sci Info"              
## [34] "Org Leader"             "Overall Plan"           "PermitsB"              
## [37] "Public Support"         "CBO Relations"
\end{verbatim}

\begin{Shaded}
\begin{Highlighting}[]
\NormalTok{Y\textless{}{-}Y[,}\KeywordTok{sort}\NormalTok{(}\KeywordTok{colnames}\NormalTok{(Y))]}

\NormalTok{combos =}\StringTok{ }\KeywordTok{data.table}\NormalTok{(}\KeywordTok{melt}\NormalTok{(}\KeywordTok{crossprod}\NormalTok{(Y)))[Var1}\OperatorTok{!=}\NormalTok{Var2,]}
\end{Highlighting}
\end{Shaded}

\begin{verbatim}
## Warning in melt(crossprod(Y)): The melt generic in data.table has been passed a
## matrix and will attempt to redirect to the relevant reshape2 method; please note
## that reshape2 is deprecated, and this redirection is now deprecated as well.
## To continue using melt methods from reshape2 while both libraries are attached,
## e.g. melt.list, you can prepend the namespace like reshape2::melt(crossprod(Y)).
## In the next version, this warning will become an error.
\end{verbatim}

\begin{Shaded}
\begin{Highlighting}[]
\NormalTok{combos}\OperatorTok{$}\NormalTok{Var1 \textless{}{-}}\StringTok{ }\KeywordTok{paste0}\NormalTok{(combos}\OperatorTok{$}\NormalTok{Var1,}\StringTok{\textquotesingle{} (\textquotesingle{}}\NormalTok{,concept\_types}\OperatorTok{$}\NormalTok{Type[}\KeywordTok{match}\NormalTok{(combos}\OperatorTok{$}\NormalTok{Var1,concept\_types}\OperatorTok{$}\NormalTok{Vector)],}\StringTok{\textquotesingle{})\textquotesingle{}}\NormalTok{)}
\NormalTok{combos}\OperatorTok{$}\NormalTok{Var2 \textless{}{-}}\StringTok{ }\KeywordTok{paste0}\NormalTok{(combos}\OperatorTok{$}\NormalTok{Var2,}\StringTok{\textquotesingle{} (\textquotesingle{}}\NormalTok{,concept\_types}\OperatorTok{$}\NormalTok{Type[}\KeywordTok{match}\NormalTok{(combos}\OperatorTok{$}\NormalTok{Var2,concept\_types}\OperatorTok{$}\NormalTok{Vector)],}\StringTok{\textquotesingle{})\textquotesingle{}}\NormalTok{)}

\NormalTok{htmlTable}\OperatorTok{::}\KeywordTok{htmlTable}\NormalTok{(combos[}\KeywordTok{order}\NormalTok{(}\OperatorTok{{-}}\NormalTok{value),][}\DecValTok{1}\OperatorTok{:}\DecValTok{10}\NormalTok{,],}\DataTypeTok{caption =}\StringTok{\textquotesingle{}top 10 combos\textquotesingle{}}\NormalTok{)}
\end{Highlighting}
\end{Shaded}

top 10 combos

Var1

Var2

value

1

Transpo (Concern)

Stormwater (Concern)

251

2

Stormwater (Concern)

Transpo (Concern)

251

3

Transpo (Concern)

DACsC (Concern)

188

4

DACsC (Concern)

Transpo (Concern)

188

5

Transpo (Concern)

SLR Plan (Policy)

184

6

SLR Plan (Policy)

Transpo (Concern)

184

7

Transpo (Concern)

Overall Plan (Barrier)

176

8

Overall Plan (Barrier)

Transpo (Concern)

176

9

Stormwater (Concern)

Overall Plan (Barrier)

166

10

Overall Plan (Barrier)

Stormwater (Concern)

166

\begin{Shaded}
\begin{Highlighting}[]
\KeywordTok{require}\NormalTok{(htmlTable)}
\end{Highlighting}
\end{Shaded}

\begin{verbatim}
## Loading required package: htmlTable
\end{verbatim}

\begin{Shaded}
\begin{Highlighting}[]
\NormalTok{concept\_unipartite =}\StringTok{ }\NormalTok{(}\KeywordTok{crossprod}\NormalTok{(Y))}

\NormalTok{uni\_net =}\StringTok{ }\KeywordTok{as.network}\NormalTok{(concept\_unipartite,}\DataTypeTok{matrix.type =} \StringTok{\textquotesingle{}adjacency\textquotesingle{}}\NormalTok{,}\DataTypeTok{directed =}\NormalTok{ F,}\DataTypeTok{ignore.eval =}\NormalTok{ F,}\DataTypeTok{names.eval =} \StringTok{\textquotesingle{}cooccurence\textquotesingle{}}\NormalTok{)}

\NormalTok{uni\_net }\OperatorTok{\%v\%}\StringTok{ \textquotesingle{}Frequency\textquotesingle{}}\NormalTok{ \textless{}{-}}\StringTok{ }\KeywordTok{diag}\NormalTok{(concept\_unipartite)}

\NormalTok{uni\_net }\OperatorTok{\%v\%}\StringTok{ \textquotesingle{}Type\textquotesingle{}}\NormalTok{ \textless{}{-}}\StringTok{ }\NormalTok{concept\_types}\OperatorTok{$}\NormalTok{Type[}\KeywordTok{match}\NormalTok{(}\KeywordTok{network.vertex.names}\NormalTok{(uni\_net),concept\_types}\OperatorTok{$}\NormalTok{Vector)]}


\NormalTok{glist =}\StringTok{ }\KeywordTok{lapply}\NormalTok{(}\KeywordTok{c}\NormalTok{(}\StringTok{\textquotesingle{}Policy\textquotesingle{}}\NormalTok{,}\StringTok{\textquotesingle{}Barrier\textquotesingle{}}\NormalTok{,}\StringTok{\textquotesingle{}Concern\textquotesingle{}}\NormalTok{),}\ControlFlowTok{function}\NormalTok{(x) }\KeywordTok{get.inducedSubgraph}\NormalTok{(uni\_net,}\DataTypeTok{v =} \KeywordTok{which}\NormalTok{(uni\_net }\OperatorTok{\%v\%}\StringTok{ \textquotesingle{}Type\textquotesingle{}} \OperatorTok{==}\StringTok{ }\NormalTok{x)) }\OperatorTok{\%\textgreater{}\%}\StringTok{ }\KeywordTok{ggnetwork}\NormalTok{() }\OperatorTok{\%\textgreater{}\%}\StringTok{ }
\StringTok{         }\KeywordTok{ggplot}\NormalTok{(.,}\KeywordTok{aes}\NormalTok{(}\DataTypeTok{x =}\NormalTok{ x,}\DataTypeTok{xend =}\NormalTok{ xend,}\DataTypeTok{y =}\NormalTok{ y,}\DataTypeTok{yend =}\NormalTok{ yend,}\DataTypeTok{label =}\NormalTok{ vertex.names))}\OperatorTok{+}\StringTok{ }
\StringTok{  }\KeywordTok{geom\_edges}\NormalTok{(}\KeywordTok{aes}\NormalTok{(}\DataTypeTok{size =}\NormalTok{ cooccurence}\OperatorTok{/}\KeywordTok{nrow}\NormalTok{(Y),}\DataTypeTok{alpha =}\NormalTok{ cooccurence)) }\OperatorTok{+}
\StringTok{  }\KeywordTok{geom\_point}\NormalTok{(}\KeywordTok{aes}\NormalTok{(}\DataTypeTok{size =}\NormalTok{ Frequency}\OperatorTok{/}\KeywordTok{nrow}\NormalTok{(Y))) }\OperatorTok{+}\StringTok{ }
\StringTok{    }\KeywordTok{ggtitle}\NormalTok{(}\KeywordTok{paste}\NormalTok{(x,}\StringTok{\textquotesingle{}co{-}occurence\textquotesingle{}}\NormalTok{)) }\OperatorTok{+}\StringTok{ }
\StringTok{  }\KeywordTok{geom\_nodelabel\_repel}\NormalTok{(}\DataTypeTok{size =} \DecValTok{2}\NormalTok{) }\OperatorTok{+}\StringTok{ }\KeywordTok{theme\_map}\NormalTok{() }\OperatorTok{+}\StringTok{ }\KeywordTok{guides}\NormalTok{(}\DataTypeTok{alpha =}\NormalTok{ F,}\DataTypeTok{size =}\NormalTok{ F) }\OperatorTok{+}\StringTok{ }
\StringTok{  }\KeywordTok{scale\_size\_continuous}\NormalTok{(}\DataTypeTok{range =} \KeywordTok{c}\NormalTok{(}\FloatTok{0.1}\NormalTok{,}\DecValTok{4}\NormalTok{))}\OperatorTok{+}
\StringTok{  }\KeywordTok{theme}\NormalTok{(}\DataTypeTok{legend.position =} \KeywordTok{c}\NormalTok{(}\FloatTok{0.8}\NormalTok{,}\FloatTok{0.2}\NormalTok{))}
\NormalTok{         )}

\KeywordTok{grid.arrange}\NormalTok{(glist[[}\DecValTok{1}\NormalTok{]],glist[[}\DecValTok{2}\NormalTok{]],glist[[}\DecValTok{3}\NormalTok{]],}\DataTypeTok{ncol =} \DecValTok{2}\NormalTok{)}
\end{Highlighting}
\end{Shaded}

\includegraphics{corplot_files/figure-latex/correlation_between_responses-1.pdf}

\begin{Shaded}
\begin{Highlighting}[]
\NormalTok{soc =}\StringTok{ }\KeywordTok{as.sociomatrix}\NormalTok{(uni\_net,}\DataTypeTok{attrname =} \StringTok{\textquotesingle{}cooccurence\textquotesingle{}}\NormalTok{)}
\NormalTok{gov\_types =}\StringTok{ }\KeywordTok{c}\NormalTok{(}\StringTok{\textquotesingle{}Regional Authority\textquotesingle{}}\NormalTok{,}\StringTok{\textquotesingle{}Existing Agency\textquotesingle{}}\NormalTok{,}\StringTok{\textquotesingle{}Collab\textquotesingle{}}\NormalTok{)}
\KeywordTok{htmlTable}\NormalTok{(}\KeywordTok{do.call}\NormalTok{(cbind,}\KeywordTok{lapply}\NormalTok{(gov\_types,}\ControlFlowTok{function}\NormalTok{(x) }
\KeywordTok{as.data.table}\NormalTok{(soc[x,],}\DataTypeTok{keep.rownames =}\NormalTok{ T)[}\KeywordTok{order}\NormalTok{(}\OperatorTok{{-}}\NormalTok{V2),][}\DecValTok{1}\OperatorTok{:}\DecValTok{10}\NormalTok{,][,}\DataTypeTok{V2:=}\KeywordTok{paste0}\NormalTok{(V1,}\StringTok{\textquotesingle{} (\textquotesingle{}}\NormalTok{,}\KeywordTok{round}\NormalTok{(V2}\OperatorTok{/}\KeywordTok{sum}\NormalTok{(Y[,x]),}\DecValTok{2}\NormalTok{),}\StringTok{\textquotesingle{})\textquotesingle{}}\NormalTok{)][,}\StringTok{\textquotesingle{}V2\textquotesingle{}}\NormalTok{])),}
\DataTypeTok{header =}\NormalTok{ gov\_types,}\DataTypeTok{caption  =} \StringTok{\textquotesingle{}Top 10 choices by respondents who choose each governance preference\textquotesingle{}}\NormalTok{,}
\DataTypeTok{tfoot =} \StringTok{\textquotesingle{}\# = proportion of respondents by governance strategy who also choose item\textquotesingle{}}\NormalTok{)}
\end{Highlighting}
\end{Shaded}

Top 10 choices by respondents who choose each governance preference

Regional Authority

Existing Agency

Collab

1

Transpo (0.62)

Transpo (0.69)

Transpo (0.59)

2

DACsC (0.54)

Stormwater (0.56)

Stormwater (0.57)

3

Stormwater (0.49)

DACsC (0.51)

Overall Plan (0.45)

4

Overall Plan (0.43)

Overall Plan (0.47)

DACsC (0.42)

5

SLR Plan (0.4)

SLR Plan (0.36)

Ecosystem (0.35)

6

Ecosystem (0.33)

Political Leader (0.29)

SLR Plan (0.33)

7

Green Infra (0.29)

Ecosystem (0.27)

Financial (0.27)

8

Political Leader (0.26)

Collab (0.21)

Political Leader (0.27)

9

Financial (0.22)

Permits (0.21)

Human (0.22)

10

Human (0.22)

Vulnerable (0.21)

Permits (0.22)

\# = proportion of respondents by governance strategy who also choose
item

\begin{Shaded}
\begin{Highlighting}[]
\KeywordTok{htmlTable}\NormalTok{(}\KeywordTok{cbind}\NormalTok{(}\KeywordTok{as.data.table}\NormalTok{(soc[}\StringTok{\textquotesingle{}Regional Authority\textquotesingle{}}\NormalTok{,],}\DataTypeTok{keep.rownames =}\NormalTok{ T)[}\KeywordTok{order}\NormalTok{(}\OperatorTok{{-}}\NormalTok{V2),][}\DecValTok{1}\OperatorTok{:}\DecValTok{10}\NormalTok{,}\DecValTok{1}\NormalTok{],}
\KeywordTok{as.data.table}\NormalTok{(soc[}\StringTok{\textquotesingle{}Existing Agency\textquotesingle{}}\NormalTok{,],}\DataTypeTok{keep.rownames =}\NormalTok{ T)[}\KeywordTok{order}\NormalTok{(}\OperatorTok{{-}}\NormalTok{V2),][}\DecValTok{1}\OperatorTok{:}\DecValTok{10}\NormalTok{,}\DecValTok{1}\NormalTok{],}
\KeywordTok{as.data.table}\NormalTok{(soc[}\StringTok{\textquotesingle{}Collab\textquotesingle{}}\NormalTok{,],}\DataTypeTok{keep.rownames =}\NormalTok{ T)[}\KeywordTok{order}\NormalTok{(}\OperatorTok{{-}}\NormalTok{V2),][}\DecValTok{1}\OperatorTok{:}\DecValTok{10}\NormalTok{,}\DecValTok{1}\NormalTok{]))}
\end{Highlighting}
\end{Shaded}

V1

V1

V1

1

Transpo

Transpo

Transpo

2

DACsC

Stormwater

Stormwater

3

Stormwater

DACsC

Overall Plan

4

Overall Plan

Overall Plan

DACsC

5

SLR Plan

SLR Plan

Ecosystem

6

Ecosystem

Political Leader

SLR Plan

7

Green Infra

Ecosystem

Financial

8

Political Leader

Collab

Political Leader

9

Financial

Permits

Human

10

Human

Vulnerable

Permits

Latent class analysis (LCA) and latent trait analysis (LTA) are similar
in principle. Both can be fit to binary incidence matrices X\_nm where N
= \# of respondents and M = \# of response items. A given X\_nm value is
respondent n's 0 or 1 choice for item m. The difference between LCA and
LTA is that LCA assumes that there is a latent categorical variable
representing groups of N's (respondents), and response items are
independent conditional on group membership. In other words, the
estimated probability of choosing a given response item is conditional
on group membership. LTA essentially works in reverse -- this model
assumes that response variables (e.g., policy concepts in our case) are
represented by a D-dimensional continuous latent variable (with D
specified prior to fitting). In other words, whereas LCA groups
respondents, LTA groups responses. There's a lot of fancy approximation
math under the hood that is beyond the scope here.

``Mixture of latent trait analyzers'' (MLTA) (Gollini 2021) are sort of
a combination of LCA and LTA. Observations are assumed to come from
groups (of respondents) and response variables
(policies/concerns/barriers) are dependent upon group AND a
group-specific D-dimensional continuous latent trait variable. The
upshot is that now, respondents are conditional on group membership AND
dependence between traits (whereas an LTA assumes responses are
independent conditional on group)

The code chunk below fits an MLTAs for combinations--0 to 4 trait
dimensions and 1 to 5 respondent groups. When D = 0, the model is just
an LCA model where p(concept\textbar group). When G = 1, that means that
there are no subgroups and the best-fit model does not fit
subgroup-specific probabilities, i.e., p(concept\textbar group) =
p(concept).

\begin{Shaded}
\begin{Highlighting}[]
\CommentTok{\#\#\#\# note {-}{-} mlta takes vectors instead of single D and G values {-}{-} but pblapply is used to parallelize \#\#\#\#\#}
\NormalTok{opts =}\StringTok{ }\KeywordTok{data.table}\NormalTok{(}\KeywordTok{expand.grid}\NormalTok{(}\DataTypeTok{D =} \DecValTok{0}\OperatorTok{:}\DecValTok{4}\NormalTok{,}\DataTypeTok{G =} \DecValTok{1}\OperatorTok{:}\DecValTok{5}\NormalTok{,}\DataTypeTok{fix =} \KeywordTok{c}\NormalTok{(F,T)))}
\NormalTok{opts =}\StringTok{ }\NormalTok{opts[D}\OperatorTok{\textgreater{}}\DecValTok{0}\OperatorTok{|!}\NormalTok{fix,]}
\CommentTok{\#\#\#\# this part takes a while, so I commented out and upload an RDS at the end }\AlertTok{\#\#\#}
\CommentTok{\# require(doParallel)}
\CommentTok{\# cluster = makeCluster(4)}
\CommentTok{\# registerDoParallel(cluster)}
\CommentTok{\# clusterExport(cl = cluster,varlist = list(\textquotesingle{}opts\textquotesingle{},\textquotesingle{}Y\textquotesingle{}))}
\CommentTok{\# clusterEvalQ(cl = cluster,require(lvm4net))}
\CommentTok{\# mlta\_tests = foreach(i = 1:nrow(opts)) \%dopar\% \{mlta(X = Y, D = opts$D[i], G = opts$G[i],wfix = opts$fix[i],nstarts = 5,maxiter = 1e3)\}}
\CommentTok{\# saveRDS(mlta\_tests,\textquotesingle{}scratch/mlta\_results.rds\textquotesingle{})}

\NormalTok{mlta\_tests =}\StringTok{ }\KeywordTok{readRDS}\NormalTok{(}\StringTok{\textquotesingle{}scratch/mlta\_results.rds\textquotesingle{}}\NormalTok{)}
\NormalTok{mlta\_results =}\StringTok{ }\KeywordTok{data.table}\NormalTok{(opts,}\DataTypeTok{BIC =} \KeywordTok{sapply}\NormalTok{(mlta\_tests,}\ControlFlowTok{function}\NormalTok{(x) x}\OperatorTok{$}\NormalTok{BIC))}
\NormalTok{mlta\_results}\OperatorTok{$}\NormalTok{G\_fix =}\StringTok{ }\KeywordTok{paste0}\NormalTok{(}\StringTok{\textquotesingle{}G = \textquotesingle{}}\NormalTok{,mlta\_results}\OperatorTok{$}\NormalTok{G,}\StringTok{\textquotesingle{} (fixed slope = \textquotesingle{}}\NormalTok{,mlta\_results}\OperatorTok{$}\NormalTok{fix,}\StringTok{\textquotesingle{})\textquotesingle{}}\NormalTok{)}
\NormalTok{mlta\_results}\OperatorTok{$}\NormalTok{BIC \textless{}{-}}\StringTok{ }\KeywordTok{round}\NormalTok{(mlta\_results}\OperatorTok{$}\NormalTok{BIC)}
\NormalTok{mlta\_cast =}\StringTok{ }\KeywordTok{dcast}\NormalTok{(mlta\_results[}\KeywordTok{order}\NormalTok{(BIC)],G\_fix }\OperatorTok{\textasciitilde{}}\StringTok{ }\NormalTok{D,}\DataTypeTok{value.var =} \StringTok{\textquotesingle{}BIC\textquotesingle{}}\NormalTok{)}


\KeywordTok{htmlTable}\NormalTok{(mlta\_cast[,}\OperatorTok{{-}}\DecValTok{1}\NormalTok{],}\DataTypeTok{caption =} \StringTok{\textquotesingle{}BIC scores by MLTA specification\textquotesingle{}}\NormalTok{,}\DataTypeTok{rnames =}\NormalTok{ mlta\_cast}\OperatorTok{$}\NormalTok{G\_fix,}\DataTypeTok{header =} \KeywordTok{paste0}\NormalTok{(}\StringTok{\textquotesingle{}D = \textquotesingle{}}\NormalTok{,}\DecValTok{0}\OperatorTok{:}\DecValTok{4}\NormalTok{),}
          \DataTypeTok{tfoot =} \StringTok{\textquotesingle{}*fixed slope refers to whether slope is constant across all groups are group{-}specific\textquotesingle{}}\NormalTok{)}
\end{Highlighting}
\end{Shaded}

BIC scores by MLTA specification

D = 0

D = 1

D = 2

D = 3

D = 4

G = 1 (fixed slope = FALSE)

23729

23776

23884

24058

24327

G = 1 (fixed slope = TRUE)

23777

23883

24059

24320

G = 2 (fixed slope = FALSE)

23745

23966

24266

24685

25211

G = 2 (fixed slope = TRUE)

23847

23980

24152

24433

G = 3 (fixed slope = FALSE)

23836

24259

24757

25381

26174

G = 3 (fixed slope = TRUE)

23939

24086

24289

24567

G = 4 (fixed slope = FALSE)

23947

24565

25264

26073

27152

G = 4 (fixed slope = TRUE)

24065

24220

24465

24717

G = 5 (fixed slope = FALSE)

24078

24908

25788

26886

28059

G = 5 (fixed slope = TRUE)

24215

24398

24614

24891

*fixed slope refers to whether slope is constant across all groups are
group-specific

So, the lamest thing here is that apparently the model with 1 group and
0 latent dimensions is the best\ldots. i.e., a LCA model with no
subgroups. SAD! The fact that G = 1, D = 0 is the best fitting model
doesn't mean there are no distinctions, is just means they aren't super
strong.

That said, the 2-group, 1 dimensional model with fixed slopes (i.e., a
single item-specific slope parameter across all groups) is pretty close
though in terms of BIC, so we can play with that a bit and see what
distinctions arise. To briefly elaborate on what the intercepts and
slopes mean -- each response item is a logistic function. That function
then varies by group. Intercepts reflect the basic propensity for group
members to choose an item, in addictive log odds. A higher intercept
means group members are more likely to choose an item, and vice versa.
Slopes represent heterogeneity amongst actors who choose a given
concept. Heterogeneity is measured in terms of the latent trait
variable. In other words, heterogeneity refers to the dependenence
structure amongst items chosen by members of a given group. That said,
we are going to ignore the slopes for now.

The first thing we can look at is the posterior probability of
respondents falling into each of the two groups (latent classes)

\begin{Shaded}
\begin{Highlighting}[]
\NormalTok{index =}\StringTok{ }\KeywordTok{which}\NormalTok{(opts}\OperatorTok{$}\NormalTok{D}\OperatorTok{==}\DecValTok{1}\OperatorTok{\&}\NormalTok{opts}\OperatorTok{$}\NormalTok{G}\OperatorTok{==}\DecValTok{2}\OperatorTok{\&}\NormalTok{opts}\OperatorTok{$}\NormalTok{fix}\OperatorTok{==}\NormalTok{T)}
\NormalTok{betas =}\StringTok{ }\NormalTok{mlta\_tests[[index]]}\OperatorTok{$}\NormalTok{b}
\NormalTok{wus =}\StringTok{ }\NormalTok{mlta\_tests[[index]]}\OperatorTok{$}\NormalTok{w}
\NormalTok{group\_probs =}\StringTok{ }\NormalTok{mlta\_tests[[index]]}\OperatorTok{$}\NormalTok{z}
\KeywordTok{colnames}\NormalTok{(betas) \textless{}{-}}\StringTok{ }\KeywordTok{colnames}\NormalTok{(wus) \textless{}{-}}\StringTok{ }\KeywordTok{colnames}\NormalTok{(Y)}

\KeywordTok{ggplot}\NormalTok{() }\OperatorTok{+}\StringTok{ }\KeywordTok{geom\_histogram}\NormalTok{(}\KeywordTok{aes}\NormalTok{(}\DataTypeTok{x =}\NormalTok{group\_probs[,}\DecValTok{1}\NormalTok{]),}\DataTypeTok{binwidth =} \FloatTok{0.02}\NormalTok{) }\OperatorTok{+}\StringTok{ }\KeywordTok{theme\_bw}\NormalTok{() }\OperatorTok{+}\StringTok{ }\KeywordTok{ggtitle}\NormalTok{(}\StringTok{\textquotesingle{}p(G = g) by respondent, D = 1, G = 2, w = fixed\textquotesingle{}}\NormalTok{) }\OperatorTok{+}\StringTok{ }
\StringTok{  }\KeywordTok{ylab}\NormalTok{(}\StringTok{\textquotesingle{}\# respondents\textquotesingle{}}\NormalTok{) }\OperatorTok{+}\StringTok{ }\KeywordTok{xlab}\NormalTok{(}\StringTok{\textquotesingle{}p(G=1)\textquotesingle{}}\NormalTok{)}
\end{Highlighting}
\end{Shaded}

\includegraphics{corplot_files/figure-latex/unnamed-chunk-1-1.pdf}

By way of comparison, here's how that shakes out when G = 3. Quite a few
respondents in each corner, but also many living in the 0.40-0.60 range
between groups. Nobody in the center.

\begin{Shaded}
\begin{Highlighting}[]
\NormalTok{index2 =}\StringTok{ }\KeywordTok{which}\NormalTok{(opts}\OperatorTok{$}\NormalTok{D}\OperatorTok{==}\DecValTok{1}\OperatorTok{\&}\NormalTok{opts}\OperatorTok{$}\NormalTok{G}\OperatorTok{==}\DecValTok{3}\OperatorTok{\&}\NormalTok{opts}\OperatorTok{$}\NormalTok{fix}\OperatorTok{==}\NormalTok{T)}
\NormalTok{group\_probs =}\StringTok{ }\NormalTok{mlta\_tests[[index2]]}\OperatorTok{$}\NormalTok{z}

\KeywordTok{ggplot}\NormalTok{() }\OperatorTok{+}\StringTok{ }\KeywordTok{geom\_point}\NormalTok{(}\KeywordTok{aes}\NormalTok{(}\DataTypeTok{x =}\NormalTok{ group\_probs[,}\DecValTok{1}\NormalTok{],}\DataTypeTok{y =}\NormalTok{ group\_probs[,}\DecValTok{2}\NormalTok{]),}\DataTypeTok{pch =} \DecValTok{21}\NormalTok{) }\OperatorTok{+}\StringTok{ }\KeywordTok{theme\_bw}\NormalTok{() }\OperatorTok{+}\StringTok{ }\KeywordTok{xlab}\NormalTok{(}\StringTok{\textquotesingle{}P(G = 1)\textquotesingle{}}\NormalTok{) }\OperatorTok{+}\StringTok{ }\KeywordTok{ylab}\NormalTok{(}\StringTok{\textquotesingle{}P(G = 2)\textquotesingle{}}\NormalTok{) }\OperatorTok{+}
\StringTok{  }\KeywordTok{ggtitle}\NormalTok{(}\StringTok{\textquotesingle{}p(G = g) by respondent, D = 1, G = 3, w = fixed\textquotesingle{}}\NormalTok{)}
\end{Highlighting}
\end{Shaded}

\includegraphics{corplot_files/figure-latex/unnamed-chunk-2-1.pdf}

Anyways, we can then evaluate the intercept terms in each logistic
response function to identify which items load heavily by group.
Ignoring the model slope parameters. Here, it's clear that some items
are identified by both groups (e.g., stormwater, transportation) while
others are more excluvisely related to one and not the other (e.g.,
concern about DACs). Some are rare in both, like ``commercial'' and
``property value''. This also begins to reveal why increased
dimensionality doens't seem to help model fit very much, many items are
about as likely to be selected by one group as the other. Heuristically,
items below the dashed line are more strongly associated with Group 1,
and above are more strongly associated with Group 2.

\begin{Shaded}
\begin{Highlighting}[]
\NormalTok{beta\_dt =}\StringTok{ }\KeywordTok{data.table}\NormalTok{(}\KeywordTok{t}\NormalTok{(betas),}\DataTypeTok{keep.rownames =}\NormalTok{ T)}
\KeywordTok{ggplot}\NormalTok{(beta\_dt,}\KeywordTok{aes}\NormalTok{(}\DataTypeTok{x =} \StringTok{\textasciigrave{}}\DataTypeTok{Group 1}\StringTok{\textasciigrave{}}\NormalTok{,}\DataTypeTok{y =} \StringTok{\textasciigrave{}}\DataTypeTok{Group 2}\StringTok{\textasciigrave{}}\NormalTok{,}\DataTypeTok{label =}\NormalTok{ rn)) }\OperatorTok{+}\StringTok{ }\KeywordTok{geom\_abline}\NormalTok{(}\DataTypeTok{lty =} \DecValTok{2}\NormalTok{,}\DataTypeTok{col =} \StringTok{\textquotesingle{}grey40\textquotesingle{}}\NormalTok{) }\OperatorTok{+}
\StringTok{  }\KeywordTok{geom\_point}\NormalTok{(}\DataTypeTok{pch =} \DecValTok{21}\NormalTok{,}\DataTypeTok{col =} \StringTok{\textquotesingle{}grey50\textquotesingle{}}\NormalTok{) }\OperatorTok{+}\StringTok{ }\KeywordTok{geom\_text\_repel}\NormalTok{(}\DataTypeTok{size =} \FloatTok{2.5}\NormalTok{,}\DataTypeTok{max.overlaps =} \DecValTok{100}\NormalTok{,}\DataTypeTok{min.segment.length =} \FloatTok{0.2}\NormalTok{) }\OperatorTok{+}
\StringTok{  }\KeywordTok{xlab}\NormalTok{(}\StringTok{\textquotesingle{}intercept, group 1\textquotesingle{}}\NormalTok{) }\OperatorTok{+}\StringTok{ }\KeywordTok{ylab}\NormalTok{(}\StringTok{\textquotesingle{}intercept, group 2\textquotesingle{}}\NormalTok{) }\OperatorTok{+}\StringTok{ }\KeywordTok{theme\_bw}\NormalTok{() }\OperatorTok{+}\StringTok{ }\KeywordTok{ggtitle}\NormalTok{(}\StringTok{\textquotesingle{}Intercept estimates for item{-}response by group\textquotesingle{}}\NormalTok{,}\StringTok{\textquotesingle{}p = 1 / exp({-}beta)\textquotesingle{}}\NormalTok{)}
\end{Highlighting}
\end{Shaded}

\includegraphics{corplot_files/figure-latex/unnamed-chunk-3-1.pdf}

There are all sorts of (or at least a few) other things we could do with
the intercepts and slopes, and evaluate the same measures fit to other
combinations of G (group) and D (latent trait) values. But for now,
let's check out the correspondence between particular items within
groups. This can be measured using ``lift''. Lift refers to the
dependence between ties, in this case measured by group. So, focusing
first on group 1, we can compute and plot the lift between each pair of
response items. Independent positive responses have a lift value of 1.
Values greater than 1 imply positive dependence, and \textless{} 1
negative dependence. In other words, this is similar to fitting matching
terms in an ERGM --\textgreater{} ``how does a tie to one particular
concept affect the probability of a tie to another concept?''. Lift
values are then logged so that log-lift values \textless{} 0 imply a
negative association and \textgreater{} 0 a positive association.

\begin{Shaded}
\begin{Highlighting}[]
\NormalTok{loglift \textless{}{-}}\StringTok{ }\KeywordTok{log}\NormalTok{(}\KeywordTok{lift}\NormalTok{(mlta\_tests[[index]], }\DataTypeTok{pdGH =} \DecValTok{21}\NormalTok{))}
\KeywordTok{colnames}\NormalTok{(loglift) \textless{}{-}}\StringTok{ }\KeywordTok{colnames}\NormalTok{(Y)[}\KeywordTok{as.numeric}\NormalTok{(}\KeywordTok{colnames}\NormalTok{(loglift))]}
\KeywordTok{rownames}\NormalTok{(loglift) \textless{}{-}}\StringTok{ }\KeywordTok{colnames}\NormalTok{(Y)[}\KeywordTok{as.numeric}\NormalTok{(}\KeywordTok{rownames}\NormalTok{(loglift))]}

\NormalTok{lift\_dt =}\StringTok{ }\KeywordTok{data.table}\NormalTok{(}\KeywordTok{melt}\NormalTok{(loglift))}
\end{Highlighting}
\end{Shaded}

\begin{verbatim}
## Warning in melt(loglift): The melt generic in data.table has been passed a
## array and will attempt to redirect to the relevant reshape2 method; please note
## that reshape2 is deprecated, and this redirection is now deprecated as well.
## To continue using melt methods from reshape2 while both libraries are attached,
## e.g. melt.list, you can prepend the namespace like reshape2::melt(loglift). In
## the next version, this warning will become an error.
\end{verbatim}

\begin{Shaded}
\begin{Highlighting}[]
\KeywordTok{ggplot}\NormalTok{(}\DataTypeTok{data =}\NormalTok{ lift\_dt[Var3}\OperatorTok{==}\StringTok{\textquotesingle{}g = 1\textquotesingle{}}\NormalTok{,],}\KeywordTok{aes}\NormalTok{(}\DataTypeTok{x =}\NormalTok{ Var2,}\DataTypeTok{y =}\NormalTok{ Var1,}\DataTypeTok{fill =}\NormalTok{ value,}\DataTypeTok{group =}\NormalTok{ Var3)) }\OperatorTok{+}\StringTok{ }\KeywordTok{facet\_wrap}\NormalTok{(}\OperatorTok{\textasciitilde{}}\NormalTok{Var3) }\OperatorTok{+}\StringTok{ }\KeywordTok{geom\_tile}\NormalTok{() }\OperatorTok{+}\StringTok{ }\KeywordTok{theme\_bw}\NormalTok{() }\OperatorTok{+}\StringTok{ }
\StringTok{    }\KeywordTok{ggtitle}\NormalTok{(}\StringTok{\textquotesingle{}Log{-}lift scores for group 1\textquotesingle{}}\NormalTok{)}\OperatorTok{+}
\StringTok{  }\KeywordTok{scale\_fill\_gradient2\_tableau}\NormalTok{(}\DataTypeTok{palette =} \StringTok{\textquotesingle{}Orange{-}Blue Diverging\textquotesingle{}}\NormalTok{,}\DataTypeTok{na.value =} \StringTok{\textquotesingle{}white\textquotesingle{}}\NormalTok{) }\OperatorTok{+}\StringTok{ }\KeywordTok{theme}\NormalTok{(}\DataTypeTok{axis.text.x =} \KeywordTok{element\_text}\NormalTok{(}\DataTypeTok{angle =} \DecValTok{45}\NormalTok{,}\DataTypeTok{hjust =} \DecValTok{1}\NormalTok{))}
\end{Highlighting}
\end{Shaded}

\includegraphics{corplot_files/figure-latex/unnamed-chunk-4-1.pdf}

\begin{Shaded}
\begin{Highlighting}[]
\KeywordTok{ggplot}\NormalTok{(}\DataTypeTok{data =}\NormalTok{ lift\_dt[Var3}\OperatorTok{==}\StringTok{\textquotesingle{}g = 2\textquotesingle{}}\NormalTok{,],}\KeywordTok{aes}\NormalTok{(}\DataTypeTok{x =}\NormalTok{ Var2,}\DataTypeTok{y =}\NormalTok{ Var1,}\DataTypeTok{fill =}\NormalTok{ value,}\DataTypeTok{group =}\NormalTok{ Var3)) }\OperatorTok{+}\StringTok{ }\KeywordTok{facet\_wrap}\NormalTok{(}\OperatorTok{\textasciitilde{}}\NormalTok{Var3) }\OperatorTok{+}\StringTok{ }\KeywordTok{geom\_tile}\NormalTok{() }\OperatorTok{+}\StringTok{ }\KeywordTok{theme\_bw}\NormalTok{() }\OperatorTok{+}\StringTok{ }
\StringTok{  }\KeywordTok{ggtitle}\NormalTok{(}\StringTok{\textquotesingle{}Log{-}lift scores for group 2\textquotesingle{}}\NormalTok{)}\OperatorTok{+}
\StringTok{  }\KeywordTok{scale\_fill\_gradient2\_tableau}\NormalTok{(}\DataTypeTok{palette =} \StringTok{\textquotesingle{}Orange{-}Blue Diverging\textquotesingle{}}\NormalTok{,}\DataTypeTok{na.value =} \StringTok{\textquotesingle{}white\textquotesingle{}}\NormalTok{) }\OperatorTok{+}\StringTok{ }\KeywordTok{theme}\NormalTok{(}\DataTypeTok{axis.text.x =} \KeywordTok{element\_text}\NormalTok{(}\DataTypeTok{angle =} \DecValTok{45}\NormalTok{,}\DataTypeTok{hjust =} \DecValTok{1}\NormalTok{))}
\end{Highlighting}
\end{Shaded}

\includegraphics{corplot_files/figure-latex/unnamed-chunk-5-1.pdf} Group
2 looks relatively similar.

Finally, I didn't include the code in this rmarkdown document becuase
the models take a long time to fit, but we can back out similar results
from bipartite ERGMs. I developed a specification that constraints each
respondent to choosing no more than 3 options from each type of
category, and then we can fit matching combo terms that reflect the
probability of observing particular combos (i.e., how likely is a
two-star where a respondent chooses ``Permits'' and ``CBO relations''?).
In any case, these results are pretty noisy, and we don't see any
significant associations between a particular governance-type preference
and choosing these other concepts. I want to keep running some more
ERGMs on background to make sure I'm doing this right, but this works as
proof of concept for constraining ties and such.

\begin{Shaded}
\begin{Highlighting}[]
\NormalTok{mod =}\StringTok{ }\KeywordTok{readRDS}\NormalTok{(}\StringTok{\textquotesingle{}scratch/start\_mod.rds\textquotesingle{}}\NormalTok{)}
\NormalTok{sm =}\StringTok{ }\KeywordTok{summary}\NormalTok{(mod)}
\NormalTok{cofs =}\StringTok{ }\KeywordTok{data.table}\NormalTok{(sm}\OperatorTok{$}\NormalTok{coefficients)}
\NormalTok{cofs}\OperatorTok{$}\NormalTok{coefs =}\StringTok{ }\KeywordTok{names}\NormalTok{(mod}\OperatorTok{$}\NormalTok{coef)}
\NormalTok{cofs =}\StringTok{ }\NormalTok{cofs[}\KeywordTok{grepl}\NormalTok{(}\StringTok{\textquotesingle{}twostar\textquotesingle{}}\NormalTok{,cofs}\OperatorTok{$}\NormalTok{coefs),]}
\NormalTok{cofs}\OperatorTok{$}\NormalTok{coefs \textless{}{-}}\StringTok{ }\KeywordTok{str\_remove}\NormalTok{(cofs}\OperatorTok{$}\NormalTok{coefs,}\StringTok{\textquotesingle{}.*names}\CharTok{\textbackslash{}\textbackslash{}}\StringTok{.\textquotesingle{}}\NormalTok{)}
\NormalTok{cofs =}\StringTok{ }\NormalTok{cofs[}\OperatorTok{!}\KeywordTok{grepl}\NormalTok{(}\StringTok{\textquotesingle{}Collab Experience\textquotesingle{}}\NormalTok{,coefs),]}
\NormalTok{cofs}\OperatorTok{$}\NormalTok{group =}\StringTok{ }\KeywordTok{ifelse}\NormalTok{(}\KeywordTok{grepl}\NormalTok{(}\StringTok{\textquotesingle{}Existing\textquotesingle{}}\NormalTok{,cofs}\OperatorTok{$}\NormalTok{coefs),}\StringTok{\textquotesingle{}Existing Agency\textquotesingle{}}\NormalTok{,}\KeywordTok{ifelse}\NormalTok{(}\KeywordTok{grepl}\NormalTok{(}\StringTok{\textquotesingle{}Regional Authority\textquotesingle{}}\NormalTok{,cofs}\OperatorTok{$}\NormalTok{coefs),}\StringTok{\textquotesingle{}Regional Authority\textquotesingle{}}\NormalTok{,}\StringTok{\textquotesingle{}Collab.\textquotesingle{}}\NormalTok{))}
\NormalTok{cofs}\OperatorTok{$}\NormalTok{coefs \textless{}{-}}\StringTok{ }\KeywordTok{str\_remove}\NormalTok{(cofs}\OperatorTok{$}\NormalTok{coefs,}\StringTok{\textquotesingle{}Regional Authority}\CharTok{\textbackslash{}\textbackslash{}}\StringTok{.|}\CharTok{\textbackslash{}\textbackslash{}}\StringTok{.Regional Authority\textquotesingle{}}\NormalTok{)}
\NormalTok{cofs}\OperatorTok{$}\NormalTok{coefs \textless{}{-}}\StringTok{ }\KeywordTok{str\_remove}\NormalTok{(cofs}\OperatorTok{$}\NormalTok{coefs,}\StringTok{\textquotesingle{}Existing Agency}\CharTok{\textbackslash{}\textbackslash{}}\StringTok{.|}\CharTok{\textbackslash{}\textbackslash{}}\StringTok{.Existing Agency\textquotesingle{}}\NormalTok{)}
\NormalTok{cofs}\OperatorTok{$}\NormalTok{coefs \textless{}{-}}\StringTok{ }\KeywordTok{str\_remove}\NormalTok{(cofs}\OperatorTok{$}\NormalTok{coefs,}\StringTok{\textquotesingle{}\^{}Collab}\CharTok{\textbackslash{}\textbackslash{}}\StringTok{.|.Collab$\textquotesingle{}}\NormalTok{)}
\NormalTok{cofs =}\StringTok{ }\NormalTok{cofs[}\KeywordTok{order}\NormalTok{(Estimate),]}
\NormalTok{cofs}\OperatorTok{$}\NormalTok{coefs \textless{}{-}}\StringTok{ }\KeywordTok{fct\_inorder}\NormalTok{(cofs}\OperatorTok{$}\NormalTok{coefs)}

\KeywordTok{ggplot}\NormalTok{(}\DataTypeTok{data =}\NormalTok{ cofs) }\OperatorTok{+}\StringTok{ }
\StringTok{  }\KeywordTok{geom\_vline}\NormalTok{(}\DataTypeTok{xintercept=}\DecValTok{0}\NormalTok{,}\DataTypeTok{lty =} \DecValTok{2}\NormalTok{,}\DataTypeTok{col =} \StringTok{\textquotesingle{}grey50\textquotesingle{}}\NormalTok{) }\OperatorTok{+}\StringTok{ }
\StringTok{  }\KeywordTok{geom\_segment}\NormalTok{(}\KeywordTok{aes}\NormalTok{(}\DataTypeTok{x =}\NormalTok{ Estimate}\FloatTok{{-}1.96}\OperatorTok{*}\StringTok{\textasciigrave{}}\DataTypeTok{Std. Error}\StringTok{\textasciigrave{}}\NormalTok{,}\DataTypeTok{xend =}\NormalTok{ Estimate}\FloatTok{+1.96}\OperatorTok{*}\StringTok{\textasciigrave{}}\DataTypeTok{Std. Error}\StringTok{\textasciigrave{}}\NormalTok{,}\DataTypeTok{y =}\NormalTok{ coefs,}\DataTypeTok{yend =}\NormalTok{ coefs)) }\OperatorTok{+}
\StringTok{  }\KeywordTok{geom\_point}\NormalTok{(}\KeywordTok{aes}\NormalTok{(}\DataTypeTok{x =}\NormalTok{ Estimate,}\DataTypeTok{y =}\NormalTok{ coefs))  }\OperatorTok{+}
\StringTok{  }\KeywordTok{facet\_wrap}\NormalTok{(}\OperatorTok{\textasciitilde{}}\NormalTok{group) }\OperatorTok{+}\StringTok{ }\KeywordTok{theme\_bw}\NormalTok{() }\OperatorTok{+}\StringTok{ }
\StringTok{  }\KeywordTok{ggtitle}\NormalTok{(}\StringTok{\textquotesingle{}Mean estimates, two{-}star pairings betwen gov. type and other concepts\textquotesingle{}}\NormalTok{)}\OperatorTok{+}\StringTok{ }\KeywordTok{theme}\NormalTok{(}\DataTypeTok{axis.title.y =} \KeywordTok{element\_blank}\NormalTok{(),}\DataTypeTok{axis.ticks =} \KeywordTok{element\_blank}\NormalTok{()) }\OperatorTok{+}
\StringTok{  }\KeywordTok{xlab}\NormalTok{(}\StringTok{\textquotesingle{}Parameter estimate (95\% confidence interval\textquotesingle{}}\NormalTok{)}
\end{Highlighting}
\end{Shaded}

\includegraphics{corplot_files/figure-latex/unnamed-chunk-6-1.pdf}

\end{document}
