% Options for packages loaded elsewhere
\PassOptionsToPackage{unicode}{hyperref}
\PassOptionsToPackage{hyphens}{url}
%
\documentclass[
]{article}
\usepackage{lmodern}
\usepackage{amssymb,amsmath}
\usepackage{ifxetex,ifluatex}
\ifnum 0\ifxetex 1\fi\ifluatex 1\fi=0 % if pdftex
  \usepackage[T1]{fontenc}
  \usepackage[utf8]{inputenc}
  \usepackage{textcomp} % provide euro and other symbols
\else % if luatex or xetex
  \usepackage{unicode-math}
  \defaultfontfeatures{Scale=MatchLowercase}
  \defaultfontfeatures[\rmfamily]{Ligatures=TeX,Scale=1}
\fi
% Use upquote if available, for straight quotes in verbatim environments
\IfFileExists{upquote.sty}{\usepackage{upquote}}{}
\IfFileExists{microtype.sty}{% use microtype if available
  \usepackage[]{microtype}
  \UseMicrotypeSet[protrusion]{basicmath} % disable protrusion for tt fonts
}{}
\makeatletter
\@ifundefined{KOMAClassName}{% if non-KOMA class
  \IfFileExists{parskip.sty}{%
    \usepackage{parskip}
  }{% else
    \setlength{\parindent}{0pt}
    \setlength{\parskip}{6pt plus 2pt minus 1pt}}
}{% if KOMA class
  \KOMAoptions{parskip=half}}
\makeatother
\usepackage{xcolor}
\IfFileExists{xurl.sty}{\usepackage{xurl}}{} % add URL line breaks if available
\IfFileExists{bookmark.sty}{\usepackage{bookmark}}{\usepackage{hyperref}}
\hypersetup{
  pdftitle={Latent trait analysis},
  hidelinks,
  pdfcreator={LaTeX via pandoc}}
\urlstyle{same} % disable monospaced font for URLs
\usepackage[margin=1in]{geometry}
\usepackage{color}
\usepackage{fancyvrb}
\newcommand{\VerbBar}{|}
\newcommand{\VERB}{\Verb[commandchars=\\\{\}]}
\DefineVerbatimEnvironment{Highlighting}{Verbatim}{commandchars=\\\{\}}
% Add ',fontsize=\small' for more characters per line
\usepackage{framed}
\definecolor{shadecolor}{RGB}{248,248,248}
\newenvironment{Shaded}{\begin{snugshade}}{\end{snugshade}}
\newcommand{\AlertTok}[1]{\textcolor[rgb]{0.94,0.16,0.16}{#1}}
\newcommand{\AnnotationTok}[1]{\textcolor[rgb]{0.56,0.35,0.01}{\textbf{\textit{#1}}}}
\newcommand{\AttributeTok}[1]{\textcolor[rgb]{0.77,0.63,0.00}{#1}}
\newcommand{\BaseNTok}[1]{\textcolor[rgb]{0.00,0.00,0.81}{#1}}
\newcommand{\BuiltInTok}[1]{#1}
\newcommand{\CharTok}[1]{\textcolor[rgb]{0.31,0.60,0.02}{#1}}
\newcommand{\CommentTok}[1]{\textcolor[rgb]{0.56,0.35,0.01}{\textit{#1}}}
\newcommand{\CommentVarTok}[1]{\textcolor[rgb]{0.56,0.35,0.01}{\textbf{\textit{#1}}}}
\newcommand{\ConstantTok}[1]{\textcolor[rgb]{0.00,0.00,0.00}{#1}}
\newcommand{\ControlFlowTok}[1]{\textcolor[rgb]{0.13,0.29,0.53}{\textbf{#1}}}
\newcommand{\DataTypeTok}[1]{\textcolor[rgb]{0.13,0.29,0.53}{#1}}
\newcommand{\DecValTok}[1]{\textcolor[rgb]{0.00,0.00,0.81}{#1}}
\newcommand{\DocumentationTok}[1]{\textcolor[rgb]{0.56,0.35,0.01}{\textbf{\textit{#1}}}}
\newcommand{\ErrorTok}[1]{\textcolor[rgb]{0.64,0.00,0.00}{\textbf{#1}}}
\newcommand{\ExtensionTok}[1]{#1}
\newcommand{\FloatTok}[1]{\textcolor[rgb]{0.00,0.00,0.81}{#1}}
\newcommand{\FunctionTok}[1]{\textcolor[rgb]{0.00,0.00,0.00}{#1}}
\newcommand{\ImportTok}[1]{#1}
\newcommand{\InformationTok}[1]{\textcolor[rgb]{0.56,0.35,0.01}{\textbf{\textit{#1}}}}
\newcommand{\KeywordTok}[1]{\textcolor[rgb]{0.13,0.29,0.53}{\textbf{#1}}}
\newcommand{\NormalTok}[1]{#1}
\newcommand{\OperatorTok}[1]{\textcolor[rgb]{0.81,0.36,0.00}{\textbf{#1}}}
\newcommand{\OtherTok}[1]{\textcolor[rgb]{0.56,0.35,0.01}{#1}}
\newcommand{\PreprocessorTok}[1]{\textcolor[rgb]{0.56,0.35,0.01}{\textit{#1}}}
\newcommand{\RegionMarkerTok}[1]{#1}
\newcommand{\SpecialCharTok}[1]{\textcolor[rgb]{0.00,0.00,0.00}{#1}}
\newcommand{\SpecialStringTok}[1]{\textcolor[rgb]{0.31,0.60,0.02}{#1}}
\newcommand{\StringTok}[1]{\textcolor[rgb]{0.31,0.60,0.02}{#1}}
\newcommand{\VariableTok}[1]{\textcolor[rgb]{0.00,0.00,0.00}{#1}}
\newcommand{\VerbatimStringTok}[1]{\textcolor[rgb]{0.31,0.60,0.02}{#1}}
\newcommand{\WarningTok}[1]{\textcolor[rgb]{0.56,0.35,0.01}{\textbf{\textit{#1}}}}
\usepackage{graphicx,grffile}
\makeatletter
\def\maxwidth{\ifdim\Gin@nat@width>\linewidth\linewidth\else\Gin@nat@width\fi}
\def\maxheight{\ifdim\Gin@nat@height>\textheight\textheight\else\Gin@nat@height\fi}
\makeatother
% Scale images if necessary, so that they will not overflow the page
% margins by default, and it is still possible to overwrite the defaults
% using explicit options in \includegraphics[width, height, ...]{}
\setkeys{Gin}{width=\maxwidth,height=\maxheight,keepaspectratio}
% Set default figure placement to htbp
\makeatletter
\def\fps@figure{htbp}
\makeatother
\setlength{\emergencystretch}{3em} % prevent overfull lines
\providecommand{\tightlist}{%
  \setlength{\itemsep}{0pt}\setlength{\parskip}{0pt}}
\setcounter{secnumdepth}{-\maxdimen} % remove section numbering

\title{Latent trait analysis}
\author{}
\date{\vspace{-2.5em}}

\begin{document}
\maketitle

Memo to Kyra and Mark

I was searching around for tools that do latent network space modeling
more quickly, and found the lvm4net package
(\url{https://cran.r-project.org/web/packages/lvm4net/lvm4net.pdf}).
It's more standalone than latentnet, which integrates nicely with
statnet and ergm capabilities. But on the upside, it is \emph{very} fast
and turns out it has some really cool tools.

Firstly, it performs latent trait analysis (see
\url{https://arxiv.org/abs/1301.2167} and
\url{https://arxiv.org/abs/1905.02659}, same author as lvm4net package).
The basic idea in this case is that the concepts survey respondents
selected are essentially categorical traits. There's a lot of stats talk
and monkey business behind that, but here's the upshot\ldots{}

\#ignore this chunk, ut just loads and builds network object

\begin{Shaded}
\begin{Highlighting}[]
\NormalTok{packs =}\KeywordTok{c}\NormalTok{(}\StringTok{'tidyverse'}\NormalTok{,}\StringTok{'purrr'}\NormalTok{,}\StringTok{'data.table'}\NormalTok{,}\StringTok{'statnet'}\NormalTok{,}\StringTok{'latentnet'}\NormalTok{,}\StringTok{'bipartite'}\NormalTok{,}\StringTok{'lvm4net'}\NormalTok{,}\StringTok{'ggthemes'}\NormalTok{)}
\NormalTok{need =}\StringTok{ }\NormalTok{packs[}\OperatorTok{!}\KeywordTok{sapply}\NormalTok{(packs,}\ControlFlowTok{function}\NormalTok{(x) }\KeywordTok{suppressMessages}\NormalTok{(}\KeywordTok{require}\NormalTok{(x,}\DataTypeTok{character.only=}\NormalTok{T)))]}
\KeywordTok{sapply}\NormalTok{(need,}\ControlFlowTok{function}\NormalTok{(x) }\KeywordTok{suppressMessages}\NormalTok{(}\KeywordTok{install.packages}\NormalTok{(x,,}\DataTypeTok{type=} \StringTok{'source'}\NormalTok{)))}
\end{Highlighting}
\end{Shaded}

\begin{verbatim}
## named list()
\end{verbatim}

\begin{Shaded}
\begin{Highlighting}[]
\KeywordTok{sapply}\NormalTok{(packs[need],}\ControlFlowTok{function}\NormalTok{(x) }\KeywordTok{suppressMessages}\NormalTok{(}\KeywordTok{require}\NormalTok{(x,}\DataTypeTok{character.only=}\NormalTok{T)))}
\end{Highlighting}
\end{Shaded}

\begin{verbatim}
## named list()
\end{verbatim}

\begin{Shaded}
\begin{Highlighting}[]
\KeywordTok{library}\NormalTok{(readxl)}
\NormalTok{orig =}\StringTok{ }\NormalTok{readxl}\OperatorTok{::}\KeywordTok{read_excel}\NormalTok{(}\StringTok{'input/SLRSurvey_Full.xlsx'}\NormalTok{)}
\NormalTok{orig}\OperatorTok{$}\NormalTok{Q4[}\KeywordTok{is.na}\NormalTok{(orig}\OperatorTok{$}\NormalTok{Q4)]<-}\StringTok{'Other'}
\CommentTok{#recode anything with fewer than 10 respondents as other}
\CommentTok{# see what wed recode}
\CommentTok{#as.data.table(table(orig$Q4))[order(-N),][N<10,]}
\NormalTok{orig}\OperatorTok{$}\NormalTok{Q4[orig}\OperatorTok{$}\NormalTok{Q4 }\OperatorTok\StringTok{ }\KeywordTok{as.data.table}\NormalTok{(}\KeywordTok{table}\NormalTok{(orig}\OperatorTok{$}\NormalTok{Q4))[}\KeywordTok{order}\NormalTok{(}\OperatorTok{-}\NormalTok{N),][N}\OperatorTok{<}\DecValTok{10}\NormalTok{,]}\OperatorTok{$}\NormalTok{V1] <-}\StringTok{ 'Other'}
\NormalTok{orig}\OperatorTok{$}\NormalTok{Q4[}\KeywordTok{grepl}\NormalTok{(}\StringTok{'Other'}\NormalTok{,orig}\OperatorTok{$}\NormalTok{Q4)]<-}\StringTok{'Other'}
\NormalTok{incidence_dt =}\StringTok{ }\KeywordTok{fread}\NormalTok{(}\StringTok{'input/NewAdjacencyMatrix_ColumnRenames.csv'}\NormalTok{)}
\NormalTok{incidence_mat =}\StringTok{ }\KeywordTok{as.matrix}\NormalTok{(incidence_dt[,}\OperatorTok{-}\KeywordTok{c}\NormalTok{(}\StringTok{'ResponseId'}\NormalTok{,}\StringTok{'ResponseId_number'}\NormalTok{,}\StringTok{'DK'}\NormalTok{)])}
\KeywordTok{rownames}\NormalTok{(incidence_mat)<-incidence_dt}\OperatorTok{$}\NormalTok{ResponseId}
\CommentTok{#drop isolates}
\NormalTok{incidence_mat =}\StringTok{ }\NormalTok{incidence_mat[}\KeywordTok{rowSums}\NormalTok{(incidence_mat)}\OperatorTok{!=}\DecValTok{0}\NormalTok{,]}

\CommentTok{#create network object}
\NormalTok{bip_net =}\StringTok{ }\KeywordTok{as.network}\NormalTok{(incidence_mat,}\DataTypeTok{matrix.type =} \StringTok{'incidence'}\NormalTok{,}\DataTypeTok{bipartite =}\NormalTok{ T,}\DataTypeTok{directed =}\NormalTok{ F,}\DataTypeTok{loops =}\NormalTok{ F)}
\CommentTok{#code actor types}
\NormalTok{bip_net }\OperatorTok\StringTok{ 'Actor_Type'}\NormalTok{ <-}\StringTok{ }\NormalTok{orig}\OperatorTok{$}\NormalTok{Q4[}\KeywordTok{match}\NormalTok{(}\KeywordTok{network.vertex.names}\NormalTok{(bip_net),orig}\OperatorTok{$}\NormalTok{ResponseId)]}
\CommentTok{#code concept types}
\NormalTok{concept_types =}\StringTok{ }\KeywordTok{fread}\NormalTok{(}\StringTok{'input/Combined_VectorTypes_NoNewOther.csv'}\NormalTok{)}
\NormalTok{bip_net }\OperatorTok\StringTok{ 'Concept_Type'}\NormalTok{ <-}\StringTok{ }\NormalTok{concept_types}\OperatorTok{$}\NormalTok{Type[}\KeywordTok{match}\NormalTok{(}\KeywordTok{network.vertex.names}\NormalTok{(bip_net), concept_types}\OperatorTok{$}\NormalTok{Vector)]}
\KeywordTok{set.vertex.attribute}\NormalTok{(bip_net,}\StringTok{'Concept_Type'}\NormalTok{,}\DataTypeTok{value =} \StringTok{'Person'}\NormalTok{,}\DataTypeTok{v =} \KeywordTok{which}\NormalTok{(}\KeywordTok{is.na}\NormalTok{(bip_net }\OperatorTok\StringTok{ 'Concept_Type'}\NormalTok{)))}
\end{Highlighting}
\end{Shaded}

\begin{Shaded}
\begin{Highlighting}[]
\CommentTok{#convert to incidence matrix}
\NormalTok{Y =}\StringTok{ }\KeywordTok{as.sociomatrix}\NormalTok{(bip_net)}
\CommentTok{#run lta with 1 to 4 dimensions}
\NormalTok{dims =}\StringTok{ }\DecValTok{1}\OperatorTok{:}\DecValTok{4}
\NormalTok{d_list =}\StringTok{ }\KeywordTok{lapply}\NormalTok{(dims,}\ControlFlowTok{function}\NormalTok{(d) }\KeywordTok{lta}\NormalTok{(Y,}\DataTypeTok{D =}\NormalTok{ d))}
\KeywordTok{names}\NormalTok{(d_list) <-}\StringTok{ }\KeywordTok{paste0}\NormalTok{(}\StringTok{'D'}\NormalTok{,dims)}
\KeywordTok{print}\NormalTok{(}\KeywordTok{sapply}\NormalTok{(d_list,}\ControlFlowTok{function}\NormalTok{(x) x}\OperatorTok{$}\NormalTok{BIC))}
\end{Highlighting}
\end{Shaded}

\begin{verbatim}
## D1.BIC (G-H Quadrature correction): D2.BIC (G-H Quadrature correction): 
##                            24977.53                            25088.76 
## D3.BIC (G-H Quadrature correction): D4.BIC (G-H Quadrature correction): 
##                            25275.91                            25570.34
\end{verbatim}

Comparing BIC scores shows that 2-d is a lot better than 1 dimension or
3-4 dimensions. Thus, the results that follow assume that the
categorical 0/1 responses to the concepts are underlain by a 2-D
continuous latent variable. The first thing we can get out of this model
is a set of intercepts and slopes for logistic response functions that
connect the probability of choosing ``1'' for a particular concept as a
function of each respondents' position within the 2-d space.

\begin{Shaded}
\begin{Highlighting}[]
\NormalTok{mod =}\StringTok{ }\NormalTok{d_list[[}\DecValTok{2}\NormalTok{]]}
\CommentTok{#intercepts}
\KeywordTok{print}\NormalTok{(mod}\OperatorTok{$}\NormalTok{b)}
\end{Highlighting}
\end{Shaded}

\begin{verbatim}
##      Item 1    Item 2    Item 3    Item 4    Item 5   Item 6    Item 7
##  -0.6087304 -1.173079 -2.118472 -1.420503 -1.267788 -1.59291 -1.782609
##      Item 8    Item 9   Item 10   Item 11   Item 12   Item 13    Item 14
##  -0.9474216 -2.024345 -1.800655 -2.773816 -1.971598 -1.325555 -0.6546694
##    Item 15   Item 16   Item 17   Item 18   Item 19    Item 20   Item 21
##  -2.527557 0.5267802 -1.405382 0.2567234 -2.704893 -0.8013628 -1.539975
##    Item 22    Item 23   Item 24   Item 25   Item 26   Item 27   Item 28
##  -3.452151 -0.2397502 -2.549427 -2.893269 -1.995153 -2.071741 -2.915063
##    Item 29   Item 30   Item 31 Item 32   Item 33   Item 34   Item 35   Item 36
##  -1.541826 -1.191989 -2.542214 -2.6334 -1.143005 -1.756605 -3.113311 -2.580458
##     Item 37   Item 38   Item 39   Item 40   Item 41
##  -0.4537304 -1.817951 -1.503274 -2.078618 -1.668431
\end{verbatim}

\begin{Shaded}
\begin{Highlighting}[]
\CommentTok{#slopes}
\KeywordTok{print}\NormalTok{(mod}\OperatorTok{$}\NormalTok{w)}
\end{Highlighting}
\end{Shaded}

\begin{verbatim}
##            Item 1      Item 2       Item 3     Item 4     Item 5      Item 6
## Dim 1 -0.05455778 -0.02103478 -0.067547679 0.01042797 -0.5322390  0.01373894
## Dim 2 -0.93326330 -0.21838299  0.009719372 0.22726876  0.1501614 -0.30270928
##          Item 7     Item 8    Item 9     Item 10     Item 11     Item 12
## Dim 1 0.8485023 -0.2968877 0.1620940 -0.08788925 -0.01650485  0.17862428
## Dim 2 0.3838903  0.4200159 0.0181001  0.20267712  0.25629399 -0.02460105
##          Item 13      Item 14   Item 15    Item 16     Item 17    Item 18
## Dim 1  0.1596526 -0.141501691 0.1300562 -0.1275781 -0.13081956 -0.4520571
## Dim 2 -0.2971735  0.005676044 0.3508687 -0.5321885 -0.06814983 -0.4662872
##           Item 19    Item 20    Item 21      Item 22   Item 23     Item 24
## Dim 1 -0.10970619 -0.4337072 -0.4786699 -0.002950863 1.2282952  0.04680427
## Dim 2 -0.06109533  0.6003194  0.1768776 -0.235605920 0.2500082 -0.15295970
##           Item 25     Item 26   Item 27     Item 28   Item 29    Item 30
## Dim 1 -0.13602443 0.260648606 0.3907546 -0.03826591 0.2646802 0.08984605
## Dim 2 -0.08603677 0.005548706 0.1853602  0.28059075 0.1145346 0.31563208
##          Item 31    Item 32     Item 33    Item 34     Item 35   Item 36
## Dim 1 0.02687508 -0.2287302 -0.06633452 -0.1265691 -0.14898681 0.2071504
## Dim 2 0.06320472  0.1579134 -0.07818821 -0.4184685 -0.03455448 0.0898867
##           Item 37     Item 38     Item 39   Item 40    Item 41
## Dim 1 -0.06665446 -0.62559720 -0.06881547 0.4342047 0.09314506
## Dim 2 -0.66993357  0.07494996 -0.19499931 0.1507341 0.21492904
\end{verbatim}

\begin{Shaded}
\begin{Highlighting}[]
\KeywordTok{unique}\NormalTok{( (bip_net }\OperatorTok\StringTok{ 'Actor_Type'}\NormalTok{)[\{bip_net }\OperatorTok\StringTok{ 'Concept_Type'}\NormalTok{\}}\OperatorTok{==}\StringTok{'Person'}\NormalTok{])}
\end{Highlighting}
\end{Shaded}

\begin{verbatim}
##  [1] "Other"                                                                   
##  [2] "Education/Consulting/Research"                                           
##  [3] "No-profit organization/Non-governmental organization"                    
##  [4] "Local government (cities, counties)"                                     
##  [5] "Environmental Group"                                                     
##  [6] "Trade/Business/Industry Group"                                           
##  [7] "Regional government"                                                     
##  [8] "Environmental Special District (e.g. Park district, open space district)"
##  [9] "State government"                                                        
## [10] "Community-based organization"                                            
## [11] "Water Infrastructure Special District (e.g. irrigation district)"        
## [12] "Federal government"
\end{verbatim}

\begin{Shaded}
\begin{Highlighting}[]
\NormalTok{trait_space =}\StringTok{ }\KeywordTok{data.table}\NormalTok{(mod}\OperatorTok{$}\NormalTok{mu)}
\NormalTok{trait_space}\OperatorTok{$}\NormalTok{org_type =}\StringTok{ }\NormalTok{(bip_net }\OperatorTok\StringTok{ 'Actor_Type'}\NormalTok{)[\{bip_net }\OperatorTok\StringTok{ 'Concept_Type'}\NormalTok{\}}\OperatorTok{==}\StringTok{'Person'}\NormalTok{]}
\KeywordTok{ggplot}\NormalTok{(trait_space) }\OperatorTok{+}\StringTok{ }\KeywordTok{geom_point}\NormalTok{(}\KeywordTok{aes}\NormalTok{(}\DataTypeTok{x =}\NormalTok{ V1,}\DataTypeTok{y =}\NormalTok{ V2,}\DataTypeTok{col =}\NormalTok{ org_type),}\DataTypeTok{pch =} \DecValTok{19}\NormalTok{,}\DataTypeTok{alpha =} \FloatTok{0.6}\NormalTok{) }\OperatorTok{+}\StringTok{ }\KeywordTok{xlab}\NormalTok{(}\StringTok{'1st dimension'}\NormalTok{) }\OperatorTok{+}\StringTok{ }
\StringTok{  }\KeywordTok{ylab}\NormalTok{(}\StringTok{"2nd dimension"}\NormalTok{) }\OperatorTok{+}\StringTok{ }\KeywordTok{ggtitle}\NormalTok{(}\StringTok{'Latent trait space'}\NormalTok{) }\OperatorTok{+}\StringTok{ }\KeywordTok{theme_bw}\NormalTok{() }\OperatorTok{+}\StringTok{ }
\StringTok{  }\KeywordTok{scale_color_tableau}\NormalTok{(}\DataTypeTok{palette =}\StringTok{'Tableau 20'}\NormalTok{)}\OperatorTok{+}\StringTok{   }\KeywordTok{scale_fill_tableau}\NormalTok{(}\DataTypeTok{palette =}\StringTok{'Tableau 20'}\NormalTok{)}\OperatorTok{+}\StringTok{ }
\StringTok{  }\KeywordTok{theme}\NormalTok{(}\DataTypeTok{legend.text =} \KeywordTok{element_text}\NormalTok{(}\DataTypeTok{size =} \DecValTok{8}\NormalTok{))}
\end{Highlighting}
\end{Shaded}

\includegraphics{lta_notebook_files/figure-latex/evaluate_latent_space-1.pdf}
Doesn't seem to be a stark pattern at this point between actor type and
location. But it's tough to see acrsoss 10+ types anyways, so a simplied
lens likely sheds more light.

The next thing that is feasible is a `mixture of latent trait analyzers'
(MLTA) model; real talk, I need to reread the arxiv paper again, because
I can't explain this very well. But the tl;dr is that it incorporates a
group structure (like the clusters we've been discussing). To keep it
simple, I'll just use the 2-d model. The code also allows for different
slopes by group (for logistic response function) but I turned that off.

\begin{Shaded}
\begin{Highlighting}[]
\NormalTok{groups =}\StringTok{ }\DecValTok{1}\OperatorTok{:}\DecValTok{4}
\NormalTok{g_list =}\StringTok{ }\KeywordTok{lapply}\NormalTok{(groups,}\ControlFlowTok{function}\NormalTok{(g) }\KeywordTok{mlta}\NormalTok{(}\DataTypeTok{X =}\NormalTok{ Y,}\DataTypeTok{D =} \DecValTok{2}\NormalTok{,}\DataTypeTok{G =}\NormalTok{ g,}\DataTypeTok{wfix =}\NormalTok{ T))}
\KeywordTok{names}\NormalTok{(g_list) <-}\StringTok{ }\KeywordTok{paste0}\NormalTok{(}\StringTok{'G'}\NormalTok{, groups)}
\KeywordTok{print}\NormalTok{(}\KeywordTok{sapply}\NormalTok{(g_list,}\ControlFlowTok{function}\NormalTok{(x) x}\OperatorTok{$}\NormalTok{BIC))}
\end{Highlighting}
\end{Shaded}

\begin{verbatim}
## G1.BIC (G-H Quadrature correction): G2.BIC (G-H Quadrature correction): 
##                            25089.13                            25195.48 
## G3.BIC (G-H Quadrature correction): G4.BIC (G-H Quadrature correction): 
##                            25325.84                            25471.82
\end{verbatim}

What is not particularly promising here is that BIC scores prefer the G
= 1 option\ldots{} so not that fun. The G = 2 option is pretty close GOF
wise, so we'll continue with that. When there are multiple groups fit,
there are cool results like the probability of membership in each group
for each node:

\begin{Shaded}
\begin{Highlighting}[]
\KeywordTok{head}\NormalTok{(}\KeywordTok{round}\NormalTok{(g_list[[}\DecValTok{2}\NormalTok{]]}\OperatorTok{$}\NormalTok{z,}\DecValTok{2}\NormalTok{),}\DecValTok{20}\NormalTok{)}
\end{Highlighting}
\end{Shaded}

\begin{verbatim}
##       [,1] [,2]
##  [1,] 0.95 0.05
##  [2,] 0.99 0.01
##  [3,] 0.04 0.96
##  [4,] 0.97 0.03
##  [5,] 1.00 0.00
##  [6,] 0.69 0.31
##  [7,] 0.80 0.20
##  [8,] 0.14 0.86
##  [9,] 1.00 0.00
## [10,] 0.96 0.04
## [11,] 0.30 0.70
## [12,] 0.01 0.99
## [13,] 0.03 0.97
## [14,] 0.86 0.14
## [15,] 0.90 0.10
## [16,] 0.02 0.98
## [17,] 0.15 0.85
## [18,] 0.99 0.01
## [19,] 1.00 0.00
## [20,] 0.97 0.03
\end{verbatim}

Since G = 2, 1 - p(g1) = p(g2). We can compare across org types:

\begin{Shaded}
\begin{Highlighting}[]
\NormalTok{p_org_dt =}\StringTok{ }\KeywordTok{data.table}\NormalTok{(g_list[[}\DecValTok{2}\NormalTok{]]}\OperatorTok{$}\NormalTok{z,}\DataTypeTok{org_type =}\NormalTok{ trait_space}\OperatorTok{$}\NormalTok{org_type)}
\KeywordTok{ggplot}\NormalTok{(p_org_dt) }\OperatorTok{+}\StringTok{ }\KeywordTok{geom_boxplot}\NormalTok{(}\KeywordTok{aes}\NormalTok{(}\DataTypeTok{y =}\NormalTok{ V1,}\DataTypeTok{x =}\NormalTok{ org_type)) }\OperatorTok{+}\StringTok{ }\KeywordTok{coord_flip}\NormalTok{() }\OperatorTok{+}\StringTok{ }\KeywordTok{theme_bw}\NormalTok{() }\OperatorTok{+}\StringTok{ }\KeywordTok{ggtitle}\NormalTok{(}\StringTok{'p(group 1) by org. type'}\NormalTok{)}
\end{Highlighting}
\end{Shaded}

\includegraphics{lta_notebook_files/figure-latex/p_group_type-1.pdf}

Just for kicks, here's G = 3\ldots{}

\begin{Shaded}
\begin{Highlighting}[]
\NormalTok{p_3group =}\StringTok{ }\KeywordTok{data.table}\NormalTok{(g_list[[}\DecValTok{3}\NormalTok{]]}\OperatorTok{$}\NormalTok{z,}\DataTypeTok{org_type =}\NormalTok{ trait_space}\OperatorTok{$}\NormalTok{org_type)}
\NormalTok{p3 =}\StringTok{ }\KeywordTok{melt}\NormalTok{(p_3group)}
\end{Highlighting}
\end{Shaded}

\begin{verbatim}
## Warning in melt.data.table(p_3group): id.vars and measure.vars are internally
## guessed when both are 'NULL'. All non-numeric/integer/logical type columns
## are considered id.vars, which in this case are columns [org_type]. Consider
## providing at least one of 'id' or 'measure' vars in future.
\end{verbatim}

\begin{Shaded}
\begin{Highlighting}[]
\NormalTok{p3}\OperatorTok{$}\NormalTok{variable <-}\StringTok{ }\KeywordTok{gsub}\NormalTok{(}\StringTok{'V'}\NormalTok{,}\StringTok{'G'}\NormalTok{,p3}\OperatorTok{$}\NormalTok{variable)}
\KeywordTok{ggplot}\NormalTok{(p3) }\OperatorTok{+}\StringTok{ }\KeywordTok{facet_wrap}\NormalTok{(}\OperatorTok{~}\NormalTok{variable,}\DataTypeTok{ncol =} \DecValTok{3}\NormalTok{) }\OperatorTok{+}\StringTok{ }\KeywordTok{geom_boxplot}\NormalTok{(}\KeywordTok{aes}\NormalTok{(}\DataTypeTok{y =}\NormalTok{ value,}\DataTypeTok{x =}\NormalTok{ org_type)) }\OperatorTok{+}\StringTok{ }
\StringTok{  }\KeywordTok{coord_flip}\NormalTok{() }\OperatorTok{+}\StringTok{ }\KeywordTok{theme_bw}\NormalTok{() }\OperatorTok{+}\StringTok{ }\KeywordTok{ggtitle}\NormalTok{(}\StringTok{'p(group) by org. type'}\NormalTok{) }\OperatorTok{+}\StringTok{ }\KeywordTok{theme}\NormalTok{(}\DataTypeTok{axis.text =} \KeywordTok{element_text}\NormalTok{(}\DataTypeTok{size =} \DecValTok{6}\NormalTok{))}
\end{Highlighting}
\end{Shaded}

\includegraphics{lta_notebook_files/figure-latex/group3-1.pdf} Instead
of the latent trait analysis above, can also fit latent class analysis
on the bipartite network. First example, 2 groups:

\begin{Shaded}
\begin{Highlighting}[]
\NormalTok{m_lca =}\StringTok{ }\KeywordTok{lca}\NormalTok{(}\DataTypeTok{X =}\NormalTok{ Y,}\DataTypeTok{G =} \DecValTok{2}\NormalTok{)}
\end{Highlighting}
\end{Shaded}

lca() outputs p-values for the conditional probability of observed a
link between a concept and an actor if the concept is in group X.

\begin{Shaded}
\begin{Highlighting}[]
\NormalTok{m_lca}\OperatorTok{$}\NormalTok{p}
\end{Highlighting}
\end{Shaded}

\begin{verbatim}
##              p_g1      p_g2      p_g3      p_g4      p_g5      p_g6       p_g7
## Group 1 0.4999666 0.2695509 0.1051539 0.1672294 0.2861284 0.2147380 0.04322116
## Group 2 0.1792811 0.1923350 0.1107415 0.2379162 0.1476491 0.1105337 0.35294105
##              p_g8      p_g9     p_g10      p_g11     p_g12     p_g13     p_g14
## Group 1 0.2751048 0.1026309 0.1323219 0.03679444 0.1155131 0.2314976 0.3744148
## Group 2 0.3029169 0.1387647 0.1589571 0.09368688 0.1342427 0.1876177 0.2965189
##              p_g15     p_g16     p_g17     p_g18      p_g19     p_g20     p_g21
## Group 1 0.01258268 0.6934768 0.2096852 0.6545647 0.07855339 0.3052323 0.2207613
## Group 2 0.17023705 0.5224705 0.1805363 0.4248463 0.04079264 0.3468897 0.1332778
##               p_g22     p_g23      p_g24      p_g25     p_g26      p_g27
## Group 1 0.048141957 0.2982611 0.09247542 0.06450384 0.1028941 0.07059298
## Group 2 0.007686697 0.6671908 0.04523959 0.03652852 0.1488472 0.18469563
##              p_g28     p_g29     p_g30      p_g31      p_g32     p_g33
## Group 1 0.02818003 0.1356376 0.1795988 0.06144027 0.06740838 0.2522755
## Group 2 0.08860390 0.2413684 0.3178076 0.08973291 0.07071699 0.2275615
##              p_g34      p_g35      p_g36     p_g37     p_g38     p_g39
## Group 1 0.21774719 0.05926941 0.04475048 0.4844129 0.2023253 0.2071062
## Group 2 0.06089387 0.01962665 0.11017352 0.2678250 0.0795168 0.1493677
##              p_g40     p_g41
## Group 1 0.06509157 0.1179822
## Group 2 0.19258269 0.2213543
\end{verbatim}

and lca() outputs p-values for the probability that each \emph{actor} is
part of group G:

\begin{Shaded}
\begin{Highlighting}[]
\KeywordTok{head}\NormalTok{(}\KeywordTok{round}\NormalTok{(m_lca}\OperatorTok{$}\NormalTok{z,}\DecValTok{2}\NormalTok{),}\DecValTok{10}\NormalTok{)}
\end{Highlighting}
\end{Shaded}

\begin{verbatim}
##       [,1] [,2]
##  [1,] 0.62 0.38
##  [2,] 0.29 0.71
##  [3,] 0.27 0.73
##  [4,] 0.87 0.13
##  [5,] 0.02 0.98
##  [6,] 0.02 0.98
##  [7,] 0.01 0.99
##  [8,] 0.05 0.95
##  [9,] 0.05 0.95
## [10,] 0.37 0.63
\end{verbatim}

\begin{Shaded}
\begin{Highlighting}[]
\NormalTok{lca_compare_G =}\StringTok{ }\KeywordTok{lca}\NormalTok{(}\DataTypeTok{X =}\NormalTok{ Y,}\DataTypeTok{G =} \DecValTok{2}\OperatorTok{:}\DecValTok{5}\NormalTok{)}
\NormalTok{lca_compare_G}\OperatorTok{$}\NormalTok{BIC}
\end{Highlighting}
\end{Shaded}

\begin{verbatim}
## $`Model Selection`
## Model with lower BIC: 
##                 "G=2" 
## 
## $`Table of BIC Results`
##      G=2      G=3      G=4      G=5 
## 24940.42 25031.81 25151.54 25316.06
\end{verbatim}

Again, G = 2 is best; G = 3 pretty close. What none of this accommodates
is the incorporation of predictive covariates as inputs. But could be
treated in relation to the \emph{outputs} (e.g., association between
covariate x and p-value of membership in one group or another).

\end{document}
